\section{Введение}
\subsection{Элементы математической логики}
Используем распространенные символы математической логики $\neg, \wedge, \vee, \Rightarrow, \Leftrightarrow$ для обозначения соответственно отрицания <<не>> и связок <<и>>, <<или>>, <<влечет>>, <<равносильно>>.


Записи $A \Rightarrow B$, означающей, что $A$ влечет $B$ или, что то же самое, $B$ следует из $A$, мы часто будем придавать другую словесную интерпретацию, говоря, что $B$ есть \textit{необходимый признак} или \textit{необходимое условие} $A$ и, в свою очередь, $A$ -- \textit{достаточное условие} или \textit{достаточный признак} $B$.
Таким образом, соотношение $A \Leftrightarrow B$ можно прочитать любым из следующих способов:
\begin{itemize}
    \item[] $A$ необходимо и достаточно для $B$;
    \item[] $A$ тогда и только тогда, когда $B$;
    \item[] $A$, если и только если $B$;
    \item[] $A$ равносильно $B$.
\end{itemize}

Некоторые логические законы:
\begin{itemize}
    \item[] $(A \Leftrightarrow B) \equiv ((A \Rightarrow B) \wedge (B \Rightarrow A))$
    \item[] $(A \Rightarrow B) \equiv (\neg A \vee B) \equiv (B \vee \neg A) \equiv (\neg (\neg B) \vee \neg A) \equiv (\neg B \Rightarrow \neg A)$
    \item[] $\neg (A \vee B) \equiv (\neg A \wedge \neg B)$
    \item[] $\neg (A \wedge B) \equiv (\neg A \vee \neg B)$
\end{itemize}

\subsection{Наивная теория множеств}
<<Под \textit{множеством} мы понимаем объединение в одно целое определенных, вполне различимых объектов нашей интуиции или нашей мысли>> -- так описал понятие «множество» Георг Кантор, основатель теории множеств. Это описание нельзя назвать определением, поскольку оно апеллирует к понятиям, ранее не определенным.


Основные предпосылки канторовской (<<наивной>>) теории множеств:
\begin{oenum}
    \item множество может состоять из любых различимых объектов;
    \item множество однозначно определяется набором составляющих его объектов;
    \item любое свойство определяет множество объектов, которые этим свойством обладают.
\end{oenum}

Если $x$ -- объект, $P$ -- свойство, $P(x)$ -- обозначение того, что x обладает свойством $P$, то через $\{ x\ \vert\ P(x) \}$ обозначают весь класс объектов, обладающих свойством $P$.
Объекты, составляющие класс или множество, называют \textit{элементами} класса или множества. Слова <<класс>>, <<семейство>>, <<совокупность>>, <<набор>> в наивной теории
множеств употребляют как синонимы термина <<множество>>.

\begin{remark}[Парадокс Рассела]
Множество всех множеств -- противоречивое понятие.
\end{remark}
\begin{proof}
Пусть $K = \left\{M  \  \vert \ P(M) \ \right\}$, где $P(M)$ означает, что $M$ не содержит себя в качестве своего элемента.

Тогда, если $K$ -- множество, то верно или $P(K)$, или $\neg P(K)$.
Действительно, $P(K)$ невозможно, так как из определения K тогда бы следовало, что $K$ содержит $K$, то есть что верно $\neg P(K)$; с другой стороны, $\neg P(K)$ тоже невозможно, поскольку это означает,
что $K$ содержит $K$, а это противоречит определению $K$ как класса тех множеств, которые сами себя не содержат.
Следовательно, $K$ -- не множество.
\end{proof}
То, что $x$ пренадлежит множеству $X$, то есть является его элементом, обозначают $x \in X$ (или $X \ni x$), а отрицание этого утверждения $x \notin X$ (или $X \not \ni x$).

Горорят, что множества \textit{равны}, если они состоят из одних и тех же элементов, то есть $A = B \ \Leftrightarrow \ \forall x ((x \in A) \Leftrightarrow (x \in B))$.
Отрицание равенства обозначают $A \not = B$.

Говорят, что $A$ является \textit{подмножеством} множества $B$, или что $B$ включает $A$, или что $B$ содержит $A$, если каждый элемент $A$ является элеметнтом множества $B$. Обозначают $A \subset B$ или $A \supset B$.
$$(A \subset B) := \forall x ((x \in A) \Rightarrow (x \in B)).$$
Если $A \subset B$ и $A \not = B$, то будем говорить, что включение $A \subset B$ \textit{строгое} или что $A$ -- \textit{собственное} подмножество $B$. Используя приведенные определения, можно заключить, что
$$(A = B) \Leftrightarrow (A \subset B) \wedge (B \subset A).$$

\textit{Пустым} называется множество, несодержащее элементов. Обозначается символом $\varnothing$. Пустое множество является подмножеством любого множества.

Основные операции над множествами:
\begin{itemize}
\item[] Пересечение множеств $(A \cap B) := \{x \ \vert \  x \in A \wedge x \in B\}$
\item[] Объединение множеств $(A \cup B) := \{x \ \vert \ x \in A \vee x \in B\}$
\item[] Разность множнеств $(A \backslash B) := \{x \ \vert \ x \in A \wedge x \notin B\}$
\item[] Симметрическая разность $(A \triangle B) := (A \backslash B) \cup (B \backslash A)$
\item[] Дизъюнктное объединение $A \sqcup B = A \cup B \ \Leftrightarrow \ A \cap B = \varnothing$
\item[] Объединение N множеств $\bigcup_{n = 1}^{N} A_n := \left\{x \ \vert \ \exists n_0 : x \in A_{n_0}\right\}$
\item[] Пересечение N множеств $\bigcap_{n = 1}^{N} A_n := \left\{x \ \vert \ \forall n \in \{1, ..., N\} : x \in A_{n}\right\}$
\end{itemize}
Пусть $\{X_\alpha\}_{\alpha \in I}$ -- семейство множеств, тогда
\begin{itemize}
\item[] Объединение семейства множеств $\bigcup_{\alpha \in I} X_\alpha := \left\{x \ \vert \ \exists \alpha \in I : x \in X_{\alpha}\right\}$
\item[] Пересечение семейства множеств $\bigcap_{\alpha \in I} X_\alpha := \left\{x \ \vert \ \forall \alpha \in I : x \in X_{\alpha}\right\}$
\end{itemize}
\begin{theorem}[Законы де Моргана]
    Пусть $X$ -- множество, $\{A_\alpha\}_{\alpha \in I}$ -- семейство множеств. Тогда
   \begin{equation}
   X \backslash \left(\bigcup_{\alpha \in I} A_\alpha \right) = \bigcap_{\alpha \in I} X \backslash A_{\alpha}
   \end{equation}
   \begin{equation}
   X \backslash \left(\bigcap_{\alpha \in I} A_\alpha \right) = \bigcup_{\alpha \in I} X \backslash A_{\alpha}
   \end{equation}
\end{theorem}
\begin{proof}
$$ X \backslash \left(\bigcup_{\alpha \in I} A_\alpha \right) = \left\{x \ \vert \ x \in X \wedge \forall \alpha \in I \ x \notin A_\alpha \right\}
= \left\{x \ \vert \ \forall \alpha \in I \ x \in X \backslash A_\alpha \right\} = \bigcap_{\alpha \in I} X \backslash A_{\alpha}$$
$$ X \backslash \left(\bigcap_{\alpha \in I} A_\alpha \right) = \left\{ x \ \vert \ x \in X \wedge \exists \alpha \in I \ x \notin A_\alpha \right\} =
\left\{x \ \vert \ \exists \alpha \in I \ x \in X \backslash A_\alpha \right\} = \bigcup_{\alpha \in I} X \backslash A_{\alpha}$$
\end{proof}
\subsection{Декартово произведение множеств. Бинарные отношения}
\begin{definition}
Пусть $X$ и $Y$ -- множества. Множество образованное всеми упорядоченными парами $(x,y)$, первый член которых есть элемент из $X$, а второй -- элемент из $Y$, называется \textit{прямым} или \textit{декартовым произведением множеств} $X$ и $Y$.
$$X \times Y := \{(x, y) \ \vert \ x \in X \wedge y \in Y\}$$
Для семейства из $n$ множеств:
$$A_1 \times A_2 \times ... \times A_n = \{(a_1, a_2, ..., a_n) \ \vert \ a_i \in A_i\}$$
\end{definition}
\begin{definition}
\textit{Бинарным отношением} $\mathcal{R}$ на множестве $X$ называют любое подмножество $X \times X$.
$$\mathcal{R} \subset X \times X$$
То что $(x, y) \in \mathcal{R}$ обозначают $x \mathcal{R} y$.
\end{definition}
Свойства бинарных отношений:
\begin{itemize}
\item[1.] $\mathcal{R}$ -- рефлексивное, если $\forall x \in X \ x \mathcal{R} x$;
\item[1'.] $\mathcal{R}$ -- иррефлексивное, если $\forall x \in X \ \neg (x \mathcal{R} x)$;
\item[2.] $\mathcal{R}$ -- симметричное, если $\forall x,y \in X \ x \mathcal{R} y \Rightarrow y \mathcal{R} x$
\item[2'.] $\mathcal{R}$ -- антисимметричное, если $\forall x,y \in X \ (x \mathcal{R} y \wedge y \mathcal{R} x) \Rightarrow x = y$
\item[3.] $\mathcal{R}$ -- транзитивно, если $\forall x, y, z \in X (x \mathcal{R} y \wedge y \mathcal{R} z) \Rightarrow x \mathcal{R} z$
\end{itemize}
\begin{definition}
Бинарное отношение, которое рефлексивно, симметрично и транзитивно, называется \textit{отношением эквивалентности}.
\end{definition}
\begin{statement}
Если бинарное отношени иррефлексивно и транзитивно, то оно и антисимметрично.
\end{statement}
\begin{proof}
O/п: Пусть $\mathcal{R}$ -- симметрично, то есть $\forall x,y \in X \ x \mathcal{R} y \Rightarrow y \mathcal{R} x$. Тогда по транзитивности
$(x \mathcal{R} y \wedge y \mathcal{R} x) \Rightarrow x \mathcal{R} x$, что противоречит с иррефлексивностью.

Следовательно, $\mathcal{R}$ -- антисимметрично.
\end{proof}

\subsection{Вещественные числа}
\begin{definition}
    Множество $\mathbb{R}$ называется множеством \textbf{вещественных чисел}, а его элементы -- \textbf{вещественными числами}, если выполнен следующий набор условий,
    называемый \textit{аксиоматикой вещественных чисел}:
\begin{itemize}
    \item[I.] Есть две бинарные операции такие, что $(\mathbb{R}, +, \cdot)$ -- \textbf{поле}, то есть выполнены \textbf{аксиомы поля}:
    \begin{itemize}
        \item[I.1] $\forall x,y,z \in \mathbb{R} \ \ x + (y + z) = (x + y) + z$
        \item[I.2] $\forall x,y \in \mathbb{R} \ \ x + y = y + x$
        \item[I.3] $\exists 0 \in \mathbb{R} : \forall x \in \mathbb{R} \ \ x + 0 = x$
        \item[I.4] $\forall x \in \mathbb{R} \ \exists (-x) \in \mathbb{R} \ : \ x + (-x) = 0$
        \item[I.5] $\forall x,y,z \in \mathbb{R} \ \ x \cdot (y \cdot z) = (x \cdot y) \cdot z$
        \item[I.6] $\forall x,y \in \mathbb{R} \ \ x \cdot y = y \cdot x$
        \item[I.7] $\exists 1 \in \mathbb{R} : \forall x \in \mathbb{R} \ \ x \cdot 1 = x$
        \item[I.8] $\forall x \in \mathbb{R} \backslash \{0\} \ \exists (x^{-1}) \in \mathbb{R} \ : \ x \cdot x^{-1} = 1$
        \item[I.9] $\forall x,y,z \in \mathbb{R} \ \ (x + y) \cdot z = x \cdot z + y \cdot z $
    \end{itemize}
    \item[II.] $\mathbb{R}$ -- \textbf{линейно упорядоченное множество}, то есть между элементами $\mathbb{R}$ определено бинарное отношение $\leq$ со следующими свойствами:
    \begin{itemize}
        \item[II.1] $\forall x,y \in \mathbb{R} \ \ ((x \leq y) \vee (y \leq x)) \equiv 1$
        \item[II.2] $\forall x \in \mathbb{R} \ \ x \leq x$
        \item[II.3] $\forall x, y \in \mathbb{R} \ \ (x \leq y \wedge  y \leq x) \Rightarrow (x = y)$
        \item[II.4] $\forall x, y, z \in \mathbb{R} \ \ (x \leq y \wedge  y \leq z) \Rightarrow (x \leq z)$
        \item[II.5] $\forall x, y, z \in \mathbb{R} \ \ x \leq y \Rightarrow x + z \leq y + z$
        \item[II.6] $0 \leq x \wedge 0 \leq y \Rightarrow 0 \leq x \cdot y$
    \end{itemize}
    \item[III.] \textbf{Аксиома полноты (непрерывности)} \\
        Если $A$ и $B$ -- непустые подмножества $\mathbb{R}$ и $\forall a \in A, \forall b \in B$ верно, что $a \leq b$, тогда $\exists c \in \mathbb{R} \ : \ a \leq c \leq b$.
        
\end{itemize}
\end{definition}
