
\documentclass[14pt, a4paper]{article}
\usepackage[utf8]{inputenc}
\usepackage[T2A]{fontenc}
\usepackage[english,russian]{babel}
\usepackage[left=25mm, top=20mm, right=20mm, bottom=20mm]{geometry}

\usepackage{amsmath,amsfonts,amssymb}
\usepackage{graphicx}

\usepackage{misccorr}%номера разделов

\usepackage[unicode]{hyperref}

\usepackage{wrapfig}

\everymath{\displaystyle}

\usepackage{asymptote}

\setlength{\headheight}{15pt}
\linespread{1.5}

\usepackage[normalem]{ulem}%зачеркнутый текст

\usepackage{amsthm}%оформление теорем
\newtheorem{theorem}{Теорема}[section]
\newtheorem{corollary}{Следствие}[theorem]
\theoremstyle{definition}
\newtheorem{definition}{Определение}[section]
\newtheorem{lemma}[theorem]{Лемма}
\newtheorem*{statement}{Утверждение}
\theoremstyle{remark}
\newtheorem*{remark}{Замечание}

%Настройка списков
\usepackage{enumitem}
\newenvironment{oenum}%
{\begin{enumerate}[label=\arabic*$^\circ$]}{\end{enumerate}}

\hypersetup{colorlinks=true,linkcolor=blue,filecolor=magenta, urlcolor=cyan}

\renewcommand\qedsymbol{$\blacksquare$}

\usepackage{fancybox,fancyhdr} %колонтитулы
\pagestyle{fancy}%нумерация страниц
\fancyhead{}
\setcounter{page}{1}
\fancyhead[L]{Математический анализ I}

\begin{document}
\hfill \break
\hfill \break
\hfill \break
\hfill \break
\hfill \break
\hfill \break
\begin{center}
\Huge{Математический анализ} \\[2cm]
\LARGE{1 семестр} \\
\end{center}
\hfill \break
\begin{center} \today \end{center}
\thispagestyle{empty}
\newpage
\tableofcontents
\thispagestyle{empty}
\newpage
\selectlanguage{Russian}

\input{introduction.tex}

\begin{thebibliography}{2}
\bibitem{Gromov} \textbf{Виноградов О.Л., Громов А. Л.} Курс математического анализа: В 5 частях.
\bibitem{Zorich} \textbf{Зорич В. А.} Математический анализ.
\bibitem{Fikhten} \textbf{Фихтенгольц Г. М.} Курс дифференциального и интегрального исчисления: В 3 т.
\bibitem{Hrabrov} \href{https://stepik.org/course/716/info}{Математический анализ(stepic)}
\end{thebibliography}
\end{document}

